\documentclass[11pt,a4paper]{article}

\usepackage[slovene]{babel}
\usepackage[utf8x]{inputenc}
\usepackage{graphicx}
\usepackage{hyperref}
\usepackage{pdfpages}

\pagestyle{plain}

\begin{document}
\title{Poročilo pri predmetu \\
Analiza podatkov s programom R}
\author{Špela Dugar}
\maketitle

\section{Izbira teme}

\subsection{PREDSTAVITEV TEME:}
Za naslov mojega projekta sem si izbrala: Rodnost oz število živorojenih otrok v Sloveniji in primerjava z drugimi članicami Evropske Unije.
V projektu bom najprej natančneje analizirala število živorojenih otrok v Sloveniji. V tabeli bom navedla naslednje podatke, za zadnje 4 leta (od leta 2010 do leta 2013):
\begin{itemize}
\item{ime regije (imenska spremenljivka)}
\item{št. živorojenih otrok v posamezni regiji (številska spremenljivka)}
\item{ katerega rojstva je bilo največ; prvo, drugo, tretje, četrto ali več (urejenostna spremenljivka)}
\item{število četrtega ali višjega rojstva (številska spremenljivka)}
\item{št. živorojenih otrok v zakonski zvezi (številska spremenljivka)}
\item{katerega rojstva je bilo največ v zakonski zvezi (urejenostna spremenljivka)}
\item{št. živorojenih otrok izven zakonske zveze (številska spremenljivka)}
\item{katerega rojstva je bilo največ izven zakonske zveze (urejenostna spremenljivka)}
\end{itemize}
Kasneje pa bom analizirala še podatke glede števila živorojenih otrok v vseh članicah Evropske Unije od leta 2003 do leta 2013. V tabeli bom navedla naslednje podatke:
\begin{itemize}
\item{ime države}
\item{št. živorojenih otrok v določenem letu}

 

\subsection{CILJI:}
V projektu bom najprej na podlagi zgornjih podatkov za Slovenijo ugotavljala, katera regija je imela največ rojstev v določenem letu, in katera v zadnjih 4 letih skupaj in navedla tudi število le-teh. Izračunala bom maksimalno in minimalno število rojstev za posamezno regijo ter povprečno število rojstev v zadnjih štirih letih za posamezno regijo. Posamezne regije bom primerjala med seboj tudi po tem, katerega rojstva je bilo največ (prvega, drugega, tretjega, četrtega ali več) in tako pridobila podatke o tem, v kateri regiji je bilo največ novopečenih mamic. Na podlagi podatka o število četrtega rojstva (ali več) pa bom ugotovila, kje v Sloveniji je največ mamic s štirimi ali več otroki.
Regije bom na nato primerjala še po tem, koliko otrok se je rodilo v zakonski zvezi (izven zakonske zveze) in katerega rojstva je bilo največ v določenem letu.
Na koncu pa bom na podlagi tabele o številu živorojenih otrok v članicah Evropske Unije ugotovila, v kateri državi je bilo največ rojstev, izračunala maksimalno in minimalno ter povprečno število rojstev v posamezni državi v zadnjih desetih letih, ter ugotovila koliko novih otrok se je rodilo v Evropski Uniji v posameznem letu.
Dobljene rezultate bom predstavila tudi na zemljevidu.

 

\subsection{PODATKI:}
Podatke za moj projekt sem pridobila na spletni strani Statističnega urada Republike Slovenije (1tabela) in EUROSTAT-a (2tabela). Z namenom, da spoznam več načinov uvoza podatkov v program R bom prvo tabelo uvozila v HTML obliki, drugo tabelo pa v CVS obliki.
Linki:
\begin{enumerate}
\item{\url{http://pxweb.stat.si/pxweb/Dialog/varval.asp?ma=05J2012S&ti=&path=../Database/Dem_soc/05_prebivalstvo/30_Rodnost/10_05J20_rojeni_RE_OBC/&lang=2}}
\item{\url{http://epp.eurostat.ec.europa.eu/tgm/table.do?tab=table&init=1&plugin=1&language=en&pcode=tps00111}}
\end{enumerate}


 
\newpage
\section{Obdelava, uvoz in čiščenje podatkov}

Uvozila sem 2 tabeli. Prvo tabelo sem dobila iz statističnega urada in jo uvozila v CSV obliki. Drugo tabelo sem dobila iz eurostat-a in jo uvozila kot xml.
Pri uvozu tabele iz eurostata sem naletela na težavo, in sicer dobila sem tabelo s samimi NA-ji. Problem sem rešila tako, da sem iz tabele pobrala vsa števila, ostalo pa izbrisala. Imena stolpcev(ki so ubistvu kar leta) sem tako podala eksplicitno.

Ko sem uvozila obe tabeli, sem se lotila z risanjem grafov. Najprej sem v mapi slike, ustvarila novo R skripto, v kateri sem v prvi in zadnji vrstici podala ukaze, ki mi uvozijo grafe v pdf obliko. Potem pa sem pričela z risanjem grafov.

Za prvo tabelo sem narisala graf, ki prikazuje število živorojenih otrok po regijah (Pomurska, Podravksa, Koroška, Savinjska, Zasavska, Spodnjeposavska, Jugovzhodna Slovenija, osrednjeslovenska, Gorenjska,Notranjsko-kraška, Gorška inObalno-kraška) za leto 2013. Prvi stolpec pa prikazuje število vseh živorojenih otrok v Sloveniji  v tem letu.

Za drugo tabelo pa sem narisala graf, ki prikazuje število vseh živorojenih otrok v držvah EU(28) in sicer od leta 2002 do leta 2012.

oba grafa sem naredila z ukazom barplot-to se pravi, da sem izbrala stolpični prikaz, saj se mi zdi najbolj primeren in pregleden za prikaz teh podatkov.

\includepdf[pages={1-2}]{../slike/grafi.pdf}

% \section{Analiza in vizualizacija podatkov}
% 
% \includegraphics{../slike/povprecna_druzina.pdf}
% 
% \section{Napredna analiza podatkov}
% 
% \includegraphics{../slike/naselja.pdf}

\end{document}
