\documentclass[11pt,a4paper]{article}

\usepackage[slovene]{babel}
\usepackage[utf8x]{inputenc}
\usepackage{graphicx}
\usepackage{hyperref}
\usepackage{pdfpages}

\pagestyle{plain}

\begin{document}
\title{Poročilo pri predmetu \\
Analiza podatkov s programom R}
\author{Špela Dugar}
\maketitle

\section{Izbira teme}

\subsection{PREDSTAVITEV TEME:}
Za naslov mojega projekta sem si izbrala: Rodnost oz število živorojenih otrok v Sloveniji in primerjava z drugimi članicami Evropske Unije.
V projektu bom najprej natančneje analizirala število živorojenih otrok v Sloveniji. V tabeli bom navedla naslednje podatke, za zadnjih 11 let (od leta 2002 do leta 2013):
\begin{itemize}
\item{ime regije (imenska spremenljivka)}
\item{št. živorojenih otrok v posamezni regiji (številska spremenljivka)}
\item{štvilo prvega rojstva, drugega, tretjega, četrtega in več}
\item{št. živorojenih otrok v zakonski zvezi (številska spremenljivka)}
\item{število prvega rojstva, drugega, tretjega, četrtega in več; v zakonski zvezi}
\item{št. živorojenih otrok izven zakonske zveze (številska spremenljivka)}
\item{število prvega rojsteva, drugega, tretjega, četrtega in ve; izven zakonske zvezeč}
\end{itemize}
Kasneje pa bom analizirala še podatke glede števila živorojenih otrok v vseh članicah Evropske Unije od leta 2003 do leta 2013. V tabeli bom navedla naslednje podatke:
\begin{itemize}
\item{ime države}
\item{št. živorojenih otrok v določenem letu}
\end{itemize}

 

\subsection{CILJI:}
V projektu bom na podlagi podatkov za Slovenijo ugotavljala, katera regija je imela največ rojstev v letu 2013 (za leto 2014 podatkov trenutno še ni na voljo). Za posamezno regijo pa bom grafično poskušala prikazati kako se je število živorojenih otrok spreminjalo skozi leta (od leta 2002 do 2013).
Na podlagi podatkov za Evropo pa bom poskušala ugotoviti katerega leta se je v Evropi rodilo največ otrok. Prikazala pa bom tudi spreminjanje števila rojstev v zadnjih 11 letih za posamezno državo.
Glavni cilj pa je predvsem spoznati orodja analiziranja v programu R na nekem konkretnem, zame osebno zanimivem primeru.


 

\subsection{PODATKI:}
Podatke za moj projekt sem pridobila na spletni strani Statističnega urada Republike Slovenije (1tabela) in EUROSTAT-a (2tabela). Z namenom, da spoznam več načinov uvoza podatkov v program R bom prvo tabelo uvozila v CSV obliki, drugo tabelo pa v XML obliki.
Linki:
\begin{enumerate}
\item{\url{http://pxweb.stat.si/pxweb/Dialog/varval.asp?ma=05J2012S&ti=&path=../Database/Dem_soc/05_prebivalstvo/30_Rodnost/10_05J20_rojeni_RE_OBC/&lang=2}}
\item{\url{http://epp.eurostat.ec.europa.eu/tgm/table.do?tab=table&init=1&plugin=1&language=en&pcode=tps00111}}
\end{enumerate}


 
\newpage
\section{Obdelava, uvoz in čiščenje podatkov}

Uvozila sem 2 tabeli. Prvo tabelo sem dobila iz statističnega urada in jo uvozila v CSV obliki. Drugo tabelo sem dobila iz eurostat-a in jo uvozila kot xml. Prvo tabelo sem v R Studiu poimenovala kot RodnostSLO, drugo tabelo pa RodnostEU.
Pri uvozu tabele iz eurostata sem naletela na težavo, in sicer dobila sem tabelo s samimi NA-ji. Problem sem rešila tako, da sem iz tabele pobrala vsa števila, ostalo pa izbrisala. Imena stolpcev(ki so ubistvu kar leta) sem tako podala eksplicitno.

Ko sem uvozila obe tabeli, sem se lotila z risanjem grafov. Najprej sem v mapi slike, ustvarila novo R skripto, v kateri sem v prvi in zadnji vrstici podala ukaze, ki mi uvozijo grafe v pdf obliko. Potem pa sem pričela z risanjem grafov.

Za prvo tabelo sem narisala graf, ki prikazuje število živorojenih otrok po regijah (Pomurska, Podravksa, Koroška, Savinjska, Zasavska, Spodnjeposavska, Jugovzhodna Slovenija, osrednjeslovenska, Gorenjska, Notranjsko-kraška, Gorška inObalno-kraška) za leto 2013. 
Iz grafa vidimo,da je bilo v letu 2013 največ rojstev v Osrednji Sloveniji, najmanj pa v Zasavskem.

Za drugo tabelo pa sem narisala graf, ki prikazuje število vseh živorojenih otrok v držvah EU(28) in sicer od leta 2002 do leta 2012. Iz grafa vidimo, da je bilo največ rojstev v Evropi leta 2008 in sicer kar 5469434.

Oba grafa sem naredila z ukazom barplot-to se pravi, da sem izbrala stolpični prikaz, saj se mi zdi najbolj primeren in pregleden za prikaz teh podatkov.

\includepdf[pages={1-2}]{../slike/grafi.pdf}

\newpage
\section{Analiza in vizualizacija podatkov}



V tej fazi sem se najprej odločila, katere zemljevide bom vključila v projekt. Glede na to, da imam podatke o številu živorojenih otrok tako za Slovenijo kot tudi za države članice Evropske Unije, sem uvozila zemljevid od Slovenije in Evrope.

ZEMLJEVID SLOVENIJE prikazuje število živorojenih otrok po regijah za leto 2013. Iz Zemljevida lahko vidimo, da je največ rojstev v letu 2013 bilo v Osrednjeslovenski regiji, sledijo ji Podravska, Savinjska in Gorenjska. Najmanj rojstev pa je bilo v Zasavskem. Zdi se smiselno, saj je v osrednji Sloveniji tudi največ prebivalcev, najmanj pa jih je v Zasavskem in posledično je zato tudi tu najmanj rojstev.



\makebox[\textwidth][c]{
\includegraphics[width=1.2\textwidth]{../slike/Slovenija.pdf}
}

\newpage
ZEMLJEVID EVROPE prikazuje število živorojenih otrok za leto 2013 za posamezne države. Iz zemljevida razberemo, da ima daleč največ rojstev Turčija, potem pa ji sledijo Francija, Zruženo kraljestvo, Nemčija, Italija in Ukrajina. Čeprav iz zemljevida ni najbolj razvidno, pa ima najmanj rojstev Lihtenštajn. Slovenija je v porpečju z večino Evropskih držav.

\makebox[\textwidth][c]{
\includegraphics[width=1.2\textwidth]{../slike/EU.pdf}
}


% \section{Analiza in vizualizacija podatkov}
% 
% \includegraphics{../slike/povprecna_druzina.pdf}
% 
% \section{Napredna analiza podatkov}
% 
% \includegraphics{../slike/naselja.pdf}

\end{document}
