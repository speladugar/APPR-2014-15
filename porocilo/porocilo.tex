\documentclass[11pt,a4paper]{article}

\usepackage[slovene]{babel}
\usepackage[utf8x]{inputenc}
\usepackage{graphicx}
\usepackage{hyperref}
\usepackage{pdfpages}

\pagestyle{plain}

\begin{document}
\title{Poročilo pri predmetu \\
Analiza podatkov s programom R}
\author{Špela Dugar}
\maketitle

\section{Izbira teme}

\subsection{PREDSTAVITEV TEME:}
Za naslov mojega projekta sem si izbrala: Rodnost oz število živorojenih otrok v Sloveniji in primerjava z drugimi članicami Evropske Unije.
V projektu bom najprej natančneje analizirala število živorojenih otrok v Sloveniji. V tabeli bom navedla naslednje podatke, za zadnjih 11 let (od leta 2002 do leta 2013):
\begin{itemize}
\item{ime regije (imenska spremenljivka)}
\item{št. živorojenih otrok v posamezni regiji (številska spremenljivka)}
\item{štvilo prvega rojstva, drugega, tretjega, četrtega in več}
\item{št. živorojenih otrok v zakonski zvezi (številska spremenljivka)}
\item{število prvega rojstva, drugega, tretjega, četrtega in več; v zakonski zvezi}
\item{št. živorojenih otrok izven zakonske zveze (številska spremenljivka)}
\item{število prvega rojsteva, drugega, tretjega, četrtega in ve; izven zakonske zvezeč}
\end{itemize}
Kasneje pa bom analizirala še podatke glede števila živorojenih otrok v vseh članicah Evropske Unije od leta 2003 do leta 2013. V tabeli bom navedla naslednje podatke:
\begin{itemize}
\item{ime države}
\item{št. živorojenih otrok v določenem letu}
\end{itemize}

 

\subsection{CILJI:}
V projektu bom na podlagi podatkov za Slovenijo ugotavljala, katera regija je imela največ rojstev v letu 2013 (za leto 2014 podatkov trenutno še ni na voljo). Za posamezno regijo pa bom grafično poskušala prikazati kako se je število živorojenih otrok spreminjalo skozi leta (od leta 2002 do 2013).
Na podlagi podatkov za Evropo pa bom poskušala ugotoviti katerega leta se je v Evropi rodilo največ otrok. Prikazala pa bom tudi spreminjanje števila rojstev v zadnjih 11 letih za posamezno državo.
Glavni cilj pa je predvsem spoznati orodja analiziranja v programu R na nekem konkretnem, zame osebno zanimivem primeru.


 

\subsection{PODATKI:}
Podatke za moj projekt sem pridobila na spletni strani Statističnega urada Republike Slovenije (1tabela) in EUROSTAT-a (2tabela). Z namenom, da spoznam več načinov uvoza podatkov v program R bom prvo tabelo uvozila v CSV obliki, drugo tabelo pa v XML obliki.
Linki:
\begin{enumerate}
\item{\url{http://pxweb.stat.si/pxweb/Dialog/varval.asp?ma=05J2012S&ti=&path=../Database/Dem_soc/05_prebivalstvo/30_Rodnost/10_05J20_rojeni_RE_OBC/&lang=2}}
\item{\url{http://epp.eurostat.ec.europa.eu/tgm/table.do?tab=table&init=1&plugin=1&language=en&pcode=tps00111}}
\end{enumerate}


 
\newpage
\section{Obdelava, uvoz in čiščenje podatkov}

Uvozila sem 2 tabeli. Prvo tabelo sem dobila iz statističnega urada in jo uvozila v CSV obliki. Drugo tabelo sem dobila iz eurostat-a in jo uvozila kot xml. Prvo tabelo sem v R Studiu poimenovala kot RodnostSLO, drugo tabelo pa RodnostEU.
Pri uvozu tabele iz eurostata sem naletela na težavo, in sicer dobila sem tabelo s samimi NA-ji. Problem sem rešila tako, da sem iz tabele pobrala vsa števila, ostalo pa izbrisala. Imena stolpcev(ki so ubistvu kar leta) sem tako podala eksplicitno.

Ko sem uvozila obe tabeli, sem se lotila z risanjem grafov. Najprej sem v mapi slike, ustvarila novo R skripto, v kateri sem v prvi in zadnji vrstici podala ukaze, ki mi uvozijo grafe v pdf obliko. Potem pa sem pričela z risanjem grafov.

Za prvo tabelo sem narisala graf, ki prikazuje število živorojenih otrok po regijah (Pomurska, Podravksa, Koroška, Savinjska, Zasavska, Spodnjeposavska, Jugovzhodna Slovenija, osrednjeslovenska, Gorenjska, Notranjsko-kraška, Gorška inObalno-kraška) za leto 2013. 
Iz grafa vidimo,da je bilo v letu 2013 največ rojstev v Osrednji Sloveniji, najmanj pa v Zasavskem.

Za drugo tabelo pa sem narisala graf, ki prikazuje število vseh živorojenih otrok v držvah EU(28) in sicer od leta 2002 do leta 2012. Iz grafa vidimo, da je bilo največ rojstev v Evropi leta 2008 in sicer kar 5469434.

Oba grafa sem naredila z ukazom barplot-to se pravi, da sem izbrala stolpični prikaz, saj se mi zdi najbolj primeren in pregleden za prikaz teh podatkov.

\includepdf[pages={1-2}]{../slike/grafi.pdf}

\newpage
\section{Analiza in vizualizacija podatkov}



V tej fazi sem se najprej odločila, katere zemljevide bom vključila v projekt. Glede na to, da imam podatke o številu živorojenih otrok tako za Slovenijo kot tudi za države članice Evropske Unije, sem uvozila zemljevid od Slovenije in Evrope.

ZEMLJEVID SLOVENIJE prikazuje število živorojenih otrok po regijah za leto 2013. Iz Zemljevida lahko vidimo, da je največ rojstev v letu 2013 bilo v Osrednjeslovenski regiji, sledijo ji Podravska, Savinjska in Gorenjska. Najmanj rojstev pa je bilo v Zasavskem. Zdi se smiselno, saj je v osrednji Sloveniji tudi največ prebivalcev, najmanj pa jih je v Zasavskem in posledično je zato tudi tu najmanj rojstev.



\makebox[\textwidth][c]{
\includegraphics[width=1.2\textwidth]{../slike/Slovenija.pdf}
}

\newpage
ZEMLJEVID EVROPE prikazuje število živorojenih otrok za leto 2013 za posamezne države. Iz zemljevida razberemo, da ima daleč največ rojstev Turčija, potem pa ji sledijo Francija, Zruženo kraljestvo, Nemčija, Italija in Ukrajina. Čeprav iz zemljevida ni najbolj razvidno, pa ima najmanj rojstev Lihtenštajn. Slovenija je v porpečju z večino Evropskih držav.

\makebox[\textwidth][c]{
\includegraphics[width=1.2\textwidth]{../slike/EU.pdf}
}

\newpage

 
\section{Napredna analiza podatkov}


Za začetek naj obnovim, kaj sem ugotovila v prvih treh razah. Torej, v letu 2013 je bilo največ rojstev v Osrednjeslovenski regiji, sledijo ji Podravska, Savinjska in Gorenjska, najmanj rojstev pa je bilo  v Zasavskem, to je razvidno iz grafa:Rodnost v SLO po regijah za leto 2013 ter iz zemljevida:Rodnost v Sloveniji za leto 2013.
Poleg tega pa se ugotovila, da je bilo največ rojstev v državah članicah EU skupaj leta 2008.
Na zemljevidu Evrope pa sem za vsako državo prikazala število živorojenih otrok v letu 2013 iz katerega je razvidno da je imela daleč največ rojstev Turčija, sledijo pa ji Francija, Združeno kraljestvo, Nemčija, Italija in Ukrajina.

V četrti fazi sem se odločila, da za posamezno državo EU naredim podrobnejšo analizo glede spremembe števila prebivalstva, kjer prebivalstvo raste oz pada bodisi zaradi naravnega prirastka (stopnja rodnosti-stopnja smrtnosti) bodisi neto migracij (št.priseljenih-št.odseljenih) in tako razvrstiti države v skupine glede na to, ali prebivalstvo v posamezni državi pada ali raste in kaj je vzrok naraščanja oz padanja prebivalstva-naravni prirastek ali neto mgracije.
Poleg tega pa narediti še napoved glede števila živorojenih otrok za Slovenijo do leta 2020.

Vse potrebne podatke za 4.fazo sem našla na spletni strani Eurostata in sicer:
\begin{enumerate}
\item{\url{http://ec.europa.eu/eurostat/statistics-explained/index.php/Population_and_population_change_statistics}}
\end{enumerate}

Nato sem posamezne tabele uvozila v CVS obliki.
\newpage
Na samem začetku me je zanimalo kako se je v Evropi število prebivalstva spreminjalo skozi čas.
Podatke za prvi graf sem vzela iz tabela POPULACIJA.
Prvi graf tako prikazuje gibanje prebivalstva od leta 1960 do leta 2012 in iz njega je lepo razvidno, da je prebivasltvo skozi vsa ta leta naraščalo in lahko pričakujemo, da bo tudi v prihodnje. 

\makebox[\textwidth][c]{
\includegraphics[width=1.5\textwidth]{../slike/Populacija.pdf}
}


Potem pa me je zanimo, kaj je vzrok za naraščanje prebivastva, visok naravni prirastek ali neto migracije (2graf).

\newpage

Podatke za drugi graf sem vzela iz tabele SPREMEMBA.
Na drugem grafu je prikazana sprememba rasti prebivalstva po letih (rumena krivulja), vidimo da je sprememba, kjub manjšim nihanjem, vedno pozitivna-torej prebivastvo je skozi vsa leta naraščalo. Svetlo modra krivulja prikazuje spremembo naravnega prirastka med leti, temno modra krivulja pa prikzuje spremembo neto migracij. Iz grafa je tako razvidno, da bil do leta 1992 glavni vzrok za rast prebivalsta naravni prirastek, po letu 1992 pa se zgodi preobrat, vidimo da od tega leta dalje je prebivastvo v Evropi naraščalo bolj zaradi neto migracij kot pa naravnega prirastka.

\makebox[\textwidth][c]{
\includegraphics[width=1.5\textwidth]{../slike/Sprememba.pdf}
}


\newpage

Podatke za tretji graf sem vzela iz tabele ROJSTVO SMRTNOST.
Tretji graf pa nam prikazuje število živorojenih otrok in število umrlih v Evropi po letih od 1960 do 2012. Vidimo, da se je število živorojenih skozi leta padalo, število umrlih pa je bilo približno enako skozi vsa leta, posledično je zato naravni prirastek v zadnjih letih nizek, a še vedno pozitiven, in kot je razvidno iz 2 grafa manjši od neto migracij.


\makebox[\textwidth][c]{
\includegraphics[width=1.5\textwidth]{../slike/RojstvoSmrtnost.pdf}
}

\newpage
Podatke za četrti in peti graf sem vzela iz tabele NARAVNI PRIRASTEK MIGRACIJE.

Četrti in peti  graf prikazuje za vsako državo posebej, za koliko se je povečal oz zmanjšal naravni prirastek (roza stolpci) ter neto migracije(rumeni stolpci) za leto 2012 (do podatkov za pretekla leta nisem mogla priti, saj so nekatere države pred kratkim postala članice EU).
Dva grafa sem naredila z namenom, da so grafi bolj berljivi, kajti države na drugem grafu imajo bistveno večje število živorojenih na leto, še posebej Turčija, zato sem jih predstavila posebej.


\makebox[\textwidth][c]{
\includegraphics[width=1.5\textwidth]{../slike/Naravniprirastekmigracije.pdf}
}


\makebox[\textwidth][c]{
\includegraphics[width=1.5\textwidth]{../slike/NaravniprirastekmigracijeEU.pdf}
}
\newline
\newline
\newline
Na podlagi leta 2012 sem tako države razdelila v naslednje skupine, ki so prikazana v tabelah:



\begin{table} [h!]

\begin{center}


\begin{tabular}{|l| l| }


\hline


\textbf{Rast prebivalstva}&\textbf{Države članice EU} \\


\hline


\multicolumn{1}{|l|}{samo zarad pozitivnega naravnega prirastka:}&\multicolumn{1}{|p{8cm}|}{ Irska, Ciper, Islandija, Makedonija, Črna Gora, Turčija} \\


\hline


\multicolumn{1}{|l|}{več zaradi pozitivnega naravnega prirastka:}&\multicolumn{1}{|p{8cm}|}{Francija, Nizozemska, Slovenija,

Združeno Kraljestvo}\\


\hline


\multicolumn{1}{|l|}{več zaradi pozitivnih neto migracij:}&\multicolumn{1}{|p{8cm}|}{Belgija, Češka, Danska, Luksemburg, Malta, Slovaška, Finska, Švedska, Lihtenštajn, Norveška, Švica}\\


\hline


\multicolumn{1}{|l|}{samo zaradi pozivinih neto migracij:}&\multicolumn{1}{|l|}{Nemčija, Italija, Avstrija}\\


\hline


\end{tabular}

\end{center}

\end{table}

\newpage
\begin{table} [h]

\begin{tabular}{|l| l| }

\hline

\textbf{Padec prebivalstva}&\textbf{Države članice EU} \\


\hline


\multicolumn{1}{|l|}{samo zarad negativnega naravnega prirastka:}&\multicolumn{1}{|p{8cm}|}{ Madžarska, Romunija, Srbija} \\


\hline


\multicolumn{1}{|l|}{več zaradi negativnega naravnega prirastka:}&\multicolumn{1}{|p{8cm}|}{Bulgarija, Hrvaška}\\


\hline


\multicolumn{1}{|l|}{več zaradi negativnih neto migracij:}&\multicolumn{1}{|p{8cm}|}{Grčija, Estonija, Litva, Latvija Portugalska}\\


\hline


\multicolumn{1}{|l|}{samo zaradi pozivinih neto migracij:}&\multicolumn{1}{|l|}{Španija, Poljska}\\


\hline



\end{tabular}


\end{table}



Za konec sem se odločila, da naredim napoved glede števila živorojenih otrok v Sloveniji.

Podatke za šesti graf pa sem vzela iz tabele RODNOST SMRTNOST SLO.
Na šestem grafu sem se omejila na Slovenijo in prikazala koliko živorojenih otrok se je rodilo v posameznih letih od 2002 do 2013 in kako dobro se podatki prilegajo različnim funkcijam.
Najprej sem iz tabele RODNOST SMRTNOST SLO pobrala leta in jih shranila v vektor leto, število živorojenih pa v stzivorojeni. Nato sem za te podatke narisala graf, kjer sem dejansko število živorojenih prikazala s svtlomodrimi pikami.
Potem sem grafu dodala tri krivulje, ki se prilegajo podatkom.
Linearni model linp sem na grafu označila z roza barvo, kvadratni model kvp pa z vijolično barvo. S pomočjo ukazov linpcoefficients sem dobila naslednjo enačbo premice: $$y= -968676.949 + 492.521x,$$ kjer je y število živorojenih otrok, in x leto.
Na podoben način sem dobila tudi enačbo parabole: $$y=-4.588137e+01x^2+1.847062e+05x - 1.858726e+08,$$ kjer je enako kot prej, y število živorojenih otrok, x pa leto.

Z oranžno barvo sem narisala še model loep, za katerega sem uporabila funkcijo  loes z lokalnim prileganjem. Os x, ki predstavlja leto, sem podajšala do leta 2020, do tam torej sega moja napoved. Izračunala sem tudi ostanek, ki je povedal da se najbolj prilega model  loess , najmanj pa linearni model.

\makebox[\textwidth][c]{
\includegraphics[width=1.5\textwidth]{../slike/Stevilozivorojenih.pdf}
}



% %(sapply(list(linp, kvp, loep), function(x) sum(x$residuals^2)),
% % $8434860.1$,$5625253.3$,$590036.3$)

Za konec lahko vidimo, da se je v Sloveniji število živorojenih otrok 2002 naprej povečevalo in sicer vse do leta 2010, potem pa je število rojstev malo upadlo, za padec števila rojstev bi bila lahko vzrok tudi gospodarska kriza. Vsekakor pa ima Slovenija pozitivni naravni prirastek in najverjetnje ga bo imela tudi v prihodnje.


% \includegraphics{../slike/naselja.pdf}


\end{document}